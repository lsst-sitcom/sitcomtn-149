\section{Photometric Calibration}
\label{sec:photometric_calibration}

We have started commissioning the full photometric calibration pipeline for
Rubin Observatory, with great success so far. For testing photometric
calibration we have obtained over 150 dithered science observations in ugriz
over the Extended Chandra Deep Field-South (ECDFS) (one of the planned LSST
deep fields), and tens more in rizy over the Euclid Deep Field South
(EDFS). All of the science data has delivered seeing of $\sim0.8$ to $\sim1.5$
arcsecond seeing, which is due to the excellent work of the AOS team.  The
validation work in this document covers the ECDFS field with more complete
filter coverage.

The precision photometric calibration software used for Rubin is the Forward
Global Calibration Method (Burke, Rykoff, et al. 2018) which was used
successfully to achieve better than 2 mmag uniformity for the Dark Energy
Survey. This software has been adapted for the LSST Science Pipelines and has
been used on Hyper Suprime Cam Special Survey Program (HSC SSP) for data
releases since DR2.

The performance on HSC data has not been as good as that on DES data due to a
number of reasons, yielding repeatability and uniformity closer to the 5 mmag
level for grizy data.  First, we have had a lot of problems with HSC
backgrounds and amp-to-amp non-linearities.  Second, the HSC survey strategy
was not well suited to self calibration due to the slow slewing of the
telescope and the long time required to change a filter, leading to lots of
isolated single-band single-night surveys.  Third, we do not have detailed
throughput scans including detector-to-detector QE variations and in-situ scans
of the significant filter variations that are required for the full forward
modeling in FGCM.

Early calibration of the ComCam data is in many ways easier than that of HSC.
First of all, we have a smaller camera (9 detectors) and thus fewer variations
to have to cross-calibrate.  Second, the camera is situated in the center and
easiest to calibrate part of the focal plane.  Third, we only have one field to
calibrate across a few nights of data so far over a limited range of airmass.
Fourth, the survey strategy (multiple bands per night dithered and repeated
with overlapping filters from night to night) is well suited to
self-calibration.  On the other hand, we do not have the CBP set up yet, so we
do not have detailed filter or detector scans available for ComCam, and are
just using the LSSTCam reference filter throughputs and average detector
throughput for the LSSTCam ITL detectors.  In addition, we do not have a flat
field screen so we have had to rely on twilight flat observations for flat fielding.

\subsection{Processing Overview}

We start with the standard ISR as documented in Section~\ref{sec:isr}. While
there are a number of challenges that we have discovered with the ITL
detectors, these are mostly near the sky level, while the testing of
photometric calibration is focused on brighter stars that are less affected by
these issues. We then apply twilight flats, which we are investigating how to
make better. At the same time, we are going to have the flat field screen and
laser and projector installed prior to the commissioning of LSSTCam, so we do
not want to spend too much time worrying about specific challenges of twilight
flats which are only necessary for ComCam.

After flat fielding we find an initial point-spread function (PSF), do a star
selection based on source and psf moments that was developed for HSC
single-frame processing, and perform an initial astrometric solution and
photometric solution (with a single zero-point per detector).  The initial
astrometric solution is used to associate star observations together prior to
global photometric calibration with FGCM.  The initial photometric solution is
used for rapid analysis and prompt processing, but is not used at all for FGCM
which relies entirely on instrumental fluxes (in units of electrons) with a
minor constraint from the reference catalog.

\subsection{Global Photometric Calibration with FGCM}

All associated stars with observations with signal-to-noise greater than 10 are
input into the FGCM solution.  In addition, reference stars from The Monster
reference catalog are associated with the stars .  Only a small fraction of the
reference stars are used in the FGCM solution, sufficient to estimate an
``absolute'' calibration (trusting that The Monster is a good absolute
reference catalog).  There is additional ongoing work with absolute calibration
with respect to the CalSpec star C26202 which is not saturated in LSST images
and is fortunately contained in ECDFS that is described below.

The FGCM model constrains the atmospheric parameters per night, as well as the
absolute throughput relative to the input scans.  The standard atmosphere is
given by MODTRAN, run at the elevation of Cerro Pachon at airmass 1.2 with an
Angstrom aerosol model.  The optics and filters are all taken from
$lsst/throughputs$ version 1.9, and the detector throughput is taken from the
ITL average of the lab scans ingested into $obs\_lsst\_data$.  Note that the
detector QEs are normalized to 1.0 at 800 nm, which is certainly
greater than the true QE at this wavelength.

