\subsection{Difference Image Analysis: Transience and Variable Objects}
\label{sec:dia_transient_variable}


\subsubsection{DIA Status}

As we have started to obtain repeated science-quality images of some fields, we have begun to build coadded templates as part of the regular weekly cumulative Data Release Processings. 
These mini-DRPs also include difference image analysis (DIA) of their constituent exposures.
Using the DRP-produced templates, we have also obtained near-real-time difference images in Prompt Processing for a few exposures. 
We have not yet had the opportunity to begin tuning template generation, difference imaging, or Real/Bogus characterization of these data, so the report below provides an initial rough characterization of DIA performance. 

\subsubsection{ML Reliability and Artifact Rates}

We ran a convolutional neural network on 51$\times$51 difference, science, and template cutouts for 912k DIASources identified in the \texttt{w\_2024\_47} data release processing.  
This processing primarily includes data from extragalactic deep fields.
These DiaSources were obtained from 4252 detector-visit images, implying an average of 21 DIASources per detector or about four thousand per equivalent full LSST focal plane.
This is somewhat less than the ten thousand DIASources expected per visit and may reflect lower sensitivity due to ongoing image quality refinement and early templates.

The CNN was trained on simulated DC2 images with additional point source injection, so caution is needed when interpreting the values returned by this classifier on ComCam data.
Nevertheless,  Figure \ref{fig:reliability_hist} shows a clear separation in reliability scores and would imply roughly a 3:1 bogus:real ratio if taken at face value.
These values will be confirmed with manual inspection.
We plan to train a purpose-built classifier on larger samples of labeled ComCam data.

\begin{figure}
\includegraphics[width=\textwidth]{dia/figures/reliability_histogram.png}
\caption{Histogram of machine-learned reliability scores computed on ComCam difference images. \label{fig:reliability_hist}}
\end{figure}

\subsection{Difference imaging QA}

\begin{figure}
\includegraphics[width=\textwidth]{dia/figures/ComCam_kernel_quiver_2024112000208.png}
\caption{Quiver plot of the implied offset between the science and template images,
  calculated from the centroid of the PSF matching kernel.
  Note that the scale differs between the different panels, but that the overall pattern
  appears coherent over the focal plane, although each CCD was solved independently.
  \label{fig:diffim-quiver}}
\end{figure}

A difference imaging afterburner is run manually on Prompt Processing output to generate diagnostic plots, such as \ref{fig:diffim-quiver}.
From \ref{fig:diffim-quiver} we see the centroid of the PSF matching kernel sampled across one detector, which reveals a systematic offset between the science and template images.
A similar offset is seen across the rest of the detectors for this visit, and in other visits.
The images comprising the template used the same astrometric reference catalog as the science image in this case, but the template was constructed with the calibrate+characterizeImage pipeline while the science image was processed with calibrateImage.
These are not included in the Prompt Processing payload to save critical time during observing.

The distribution of sources detected on the difference image reveals some detector-level effects that are not fully accounted for.
Binning the locations of the diaSources in 1-D in \ref{fig:diaSource-distribution} by their x- and y-values reveals systematic overdensities of detections at the amplifier boundaries in x (but not in y).
Additional overdensities seen in y-band may be from the residual phosphorescent wax reported to be left on some chips beneath the AR coating.

\begin{figure}
\includegraphics[width=\textwidth]{dia/figures/diaSource_distribution.png}
\caption{Binning the locations of the diaSources in 1-D by their x- and y-values reveals systematic overdensities of detections at the amplifier boundaries in x (but not in y). \label{fig:diaSource-distribution}}
\end{figure}

We have analyzed the sources that we detected from the difference imaging campaign. Figure \ref{fig:diaSrc_count} shows the distribution of the diaSources as a function of the magnitude of each source when detected. We have separated the sources into six bands in which the observation was conducted. We show the growth of observed sources in each weekly processing (from week 48 to week 50). 

\begin{figure}
  \includegraphics[width=\textwidth]{dia/figures/diaSources.png}
  \caption{Number of diaSources detected during the campaign, separated in observing bands and weekly processing runs.}
  \label{fig:diaSrc_count}
\end{figure}

The number of diaSources detected increases as each weekly processing run has more data available. Having said that, the details of the distributions also change. We have verified that this is not due to differences in the algorithm details of different processing runs. This verification was done by observing the data distribution done on different weekly processing runs but taking into account only the data sets already available for weekly-48 runs. In this way, all of the runs have the same underlying data, and the differences are not so stark. \par

We want to point out two features visible in the plots. First is the steep rise and significant number of diaSources at the bright end of the distribution. We believe this is primarily due to inefficiencies in the subtraction of bright sources, making it that bright point sources on the sky generate diaSources even if they are not variable. See an example of this behavior in the left panel of image \href{https://rubin-obs.slack.com/archives/C07QM71SZ5J/p1734564452594809?thread_ts=1734547527.434299&cid=C07QM71SZ5J}{here}.
The increase at the faint end is at least partly due to noise fluctuations being detected as diaSources. See an example of this behavior in the right panel of image \href{https://rubin-obs.slack.com/archives/C07QM71SZ5J/p1734564452594809?thread_ts=1734547527.434299&cid=C07QM71SZ5J}{here}. We are still investigating other features seen in the distribution. \par

We have also investigated the differences between various fields that were observed. We did this investigation to ensure that the results are not dominated by particularly dense fields, such as 47-Tuc. The results are shown in Figure \ref{fig:diaSrc_fields}, and show that 47-Tuc is not dominating the results.

\begin{figure}
  \includegraphics[width=\textwidth]{dia/figures/diaSources_50_6_fields.png}
  \caption{Number of diaSources detected during the campaign, separated across 6 different sky locations that were repeatedly observed.}
  \label{fig:diaSrc_fields}
\end{figure}

We have also run the early implementation of the real-bogus classifier (as already discussed and seen in \ref{fig:reliability_hist}) and present the results in Figure \ref{fig:diaSrc_real_bogus}. Results are encouraging in the sense that many faint sources are flagged as likely false. 

\begin{figure}
  \includegraphics[width=\textwidth]{dia/figures/diaSource_real_bogus.png}
  \caption{Number of diaSources detected during the campaign, separated in high reliability (determined to likely be real) and low reliability (likely to be bogus) sources.}
  \label{fig:diaSrc_real_bogus}
\end{figure}


\iffalse                                % doesn't analyse comCam data.
\subsection{Cosmic Ray Detection Performance}

In this analysis, we evaluate the performance of the cosmic ray detection algorithm by integrating real cosmic ray events into simulated observational data and applying standard detection pipelines to assess their performance. The primary objective is to assess the algorithm's ability to accurately identify cosmic rays, thereby ensuring data purity for subsequent scientific analyses.

\subsubsection{Methodology}

We used 67 visits (603 detectors), where the dark frames were taken from 2024-10-27 to 2024-11-01, and the simulated exposures are from $day\_obs=20240626$.

We processed a set of simulated on-sky images from the ComCamSim OR4 repository and added real ComCam dark frames containing only cosmic rays. First, we performed Instrument Signature Removal (ISR) on ComCam darks as well as ComCamSim separately. After that, we combined the two images and performed image characterization and calibration.

To establish ground truth, we used the ComCam dark frames and identified pixels that were brighter than 5 times the standard deviation above the mean background level. This threshold allowed us to create a ground truth mask representing cosmic rays. At the end of the process, we compared this ground truth mask with the cosmic ray (CR) mask obtained after the characterization step to evaluate the algorithm's performance.

\subsubsection{Results}

The detection outcomes were categorized into four classes: True Positives (TP), False Positives (FP), True Negatives (TN), and False Negatives (FN). The counts for each category are presented in Table~\ref{tab:confusion_matrix}. The True Positives, False Positives, True Negatives, and False Negatives represent the comparison between the CR mask and the ground truth mask. Critical False Negatives (CFN), on the other hand, are the pixels that are considered detections (from the DETECTED mask) but are, in fact, cosmic rays.

\begin{table}[H]
    \centering
    \caption{Confusion Matrix for Cosmic Ray Detection (Pixel counts)}
    \label{tab:confusion_matrix}
    \begin{tabular}{*3c}
        \hline
        & \textbf{Positive (Detected)} & \textbf{Negative (Not Detected)} \\
        \hline
        \textbf{True (Cosmic Ray)}  & 223,982 & 92,366 \\
        \textbf{False (Non-Cosmic Ray)} & 86,094  & 9,821,261,558 \\
        \hline
         \multicolumn{2}{c}{\textbf{CFN (Cosmic Ray detected as Source)}} & 27954 \\
    \end{tabular}
\end{table}

From these results, we calculated the following performance metrics:

\begin{itemize}
    \item \textbf{Purity (Precision):} 
    \begin{equation*}
        \text{Purity} = \frac{TP}{TP + FP} = \frac{223,982}{223,982 + 86,094} \approx 0.722
    \end{equation*}
    \item \textbf{Completeness (Recall):} 
    \begin{equation*}
        \text{Completeness} = \frac{TP}{TP + FN} = \frac{223,982}{223,982 + 92,366} \approx 0.708
    \end{equation*}
    \item \textbf{Adjusted Completeness:} Considering critical false negatives (CFN), where cosmic rays were detected as sources, the adjusted completeness is 
    \begin{equation*}
        \text{Adjusted Completeness} = \frac{TP}{TP + CFN} = \frac{223,982}{223,982 + 27,954} \approx 0.889
    \end{equation*}
\end{itemize}

\subsubsection{Discussion}

The results indicate that the current cosmic ray detection algorithm demonstrates moderate success in identifying cosmic rays, as evidenced by a purity of approximately 72.2\%. This suggests that most of the detections are indeed true cosmic rays, though a significant portion, about 27.8\%, are false positives.

The completeness, at around 70.8\%, reveals that while the algorithm is capable of detecting a majority of cosmic rays, it still misses a substantial number of events. However, the adjusted completeness of 88.9\% highlights the algorithm's improved performance when considering critical false negatives—those cosmic rays that were detected but mistakenly classified as sources rather than cosmic rays. 


\fi

\subsection{Satellite Streaks}

As orbital space becomes increasingly crowded, we expect to see many bright streaks, flares, and glints due to
satellites and other reflective human-made objects orbiting the Earth, with the majority of the population is in low-Earth orbit (LEO).

As expected, many ComCam detector-visit images clearly show streaks. Visual inspection of nearly all ComCam images to date are being recorded on a best-effort basis in a Confluence Database dubbed ``ComCam Satellite Spotter,'' and as of 2024 Nov 25 there are over 500 rows. 
\begin{itemize}
\item Straight bright linear feature, typically at least 20 pixels wide, that crosses one or more detectors and goes off the edge (typical of most LEO satellites, such as Starlink)
\item Shorter version of the above, with clear start and/or endpoints, which usually indicates the object imaged is located at a higher-than-LEO orbital altitude (and/or the exposure integration time was unusually short)
\item Intermittent linear feature, i.e., a dashed line, due to different parts of the satellite having different reflective properties
\item A flare or glint brightening event that fades in and out along a linear trajectory, either isolated or as part of a longer streak
\item Actually a bright star diffraction spike
\item Actually a cosmic ray that was not repaired
\item Variation of any of the above but in out-of-focus donut form (interestingly, depending on altitude, certain streaks may appear either in- or out-of-focus when stars appear as donuts)
\end{itemize}

Thanks to ComCam's relatively small field of view and the satellite population being as small as it ever will
be during Rubin Commissioning and Operations, we have not yet seen an overwhelmingly bright satellite (e.g.,
BlueWalker 3 or one of the BlueBirds, all operated by AST SpaceMobile). Only a couple instances have streaks
bright enough to induce visually-obvious crosstalk ``secondary streaks;'' the majority of streaks are
relatively faint and the only portion of the image impacted are regions overlapping with the streak itself.
Reliably determining streak width is an ongoing challenge, as they are wider than the PSF, and some of
the brighter streaks have noticeably extended stray light wings.


\subsection{Fake Source Injection for DIA}
\subsubsection{Selection of a data subset}

In this subsection we select a portion of the observations processed by the DRP pipelines that include DIA, and run some simple analysis checks on the data contents. There are many fields being observed with LSSTComCam that span a large portion of the southern hemisphere accessible sky at this time of year (fall 2024).

From these, we make a selection, choosing a very well covered field in all bandpasses. In Fig.\ref{fig:deep_df} we show the full coverage of LSSTComCam up to current date.

\begin{figure}
    \centering
    \includegraphics[width=0.95\linewidth]{dia/figures/deep_df_chip_observations.png}
    \caption{The full LSSTComCam set of observations overlapping the chosen field and with elevation > 45 above the horizon.}
    \label{fig:deep_df}
    
\end{figure}

In Fig~\ref{fig:sampled_df_chip_observations} we apply a first cut on the observations, to narrow our visit list to a total of 24 visits. This cut involves a zenith distance of less than 45 degrees, as well as some technical parameters, derived from the nominal DRP DIA performance (ratio of PSF between template and science, as well as Kernel basis condition number). 

\begin{figure}
    \centering
    \includegraphics[width=0.95\linewidth]{dia/figures/sampled_df_chip_observations.png}
    \caption{The selection of CCDs for fake injection processing.}
    \label{fig:sampled_df_chip_observations}
\end{figure}


We create a catalog of fakes for this sample of visits, by creating position and magnitudes of the synthetic sources to be injected. In Fig~\ref{fig:sampled_df_chip_observations_plusfakes} we show the fake distribution per chip in our selection. We use sources that are possibly extended, and choose the fake location in its sourroundings. The magnitudes are chosen so they are within 1.5 magnitudes of the selected source host.

\begin{figure}
    \centering
    \includegraphics[width=0.95\linewidth]{dia/figures/sampled_df_chip_observations_plusfakes.png}
    \caption{The fake position distribution in sky and in the CCDs footprint per bandpass.}
    \label{fig:sampled_df_chip_observations_plusfakes}
\end{figure}

\begin{figure}
    \centering
    \includegraphics[width=0.95\linewidth]{dia/figures/simple_hist_completeness_mag_per_filter.png}
    \caption{The distribution of magnitudes per bandpass for all the injected fakes (solid lines), and in shaded region the distribution of magnitudes of the fakes detected by the AP pipeline.}
    \label{fig:found_fakes_per_Filter}
\end{figure}


We run Alert Production pipeline with a set of additional tasks that handle fake injection on the \textit{initial\_pvi} images, and then the book-keeping tasks of fake catalog matching to diaSources as well as forced photometry for SNR estimation.

We then, can compile useful statistics about the fake source recovery, that informs us about DIA algorithm performance, as well as detection and measurement algorithm performance. 

For this, we cross-match the position of our candidate detections, or \textit{diaSources} with the positions of the synthetic sources, using a tolerance of $0.5''$ (roughly $2.5$px). 


\begin{figure}
    \centering
    \includegraphics[width=0.49\linewidth]{dia/figures/scatter_xy_diaSrcs_match_px.png}
    \includegraphics[width=0.49\linewidth]{dia/figures/scatter_radec_diaSrcs_match_arsec_mag.png}
    \caption{The scatter of the coordinate centroid recovery of the fakes. In the left we have the scatter around the true centroid in pixel coordinates, and in the right the scatter around the true center of fakes in sky coordinates (and in units of arc-seconds), wit the grid matching the pixel grid by means of the platescale. Also, we include the brightness in colormap.}
    \label{fig:scatter_radec_diaSrcs_match_arsec_mag}
\end{figure}

Those fakes that found a match are called "found fake" and objects that had no match we refer as "lost" or "missed" fakes. 
The rate of found to existing fakes is our recovery rate, Recall or Efficiency of detection. 

\begin{figure}
    \centering
    \includegraphics[width=0.95\linewidth]{dia/figures/scatter_mag_ap_mag_perfilter.png}
    \caption{The recovered Aperture magnitude residual per filter for all the found fake sample, as a function of their true magnitude. }
    \label{fig:photometric_recovery_vs_fakemag}
\end{figure}
\begin{figure}
    \centering
    \includegraphics[width=0.95\linewidth]{dia/figures/scatter_mag_psf_mag_perfilter.png}
    \caption{The residual of PSF magnitude measurement for found fakes, as function of their true magnitude.}
    \label{fig:photometric_recovery_psf_vs_fakemag}
\end{figure}
\begin{figure}
    \centering
    \includegraphics[width=0.95\linewidth]{dia/figures/scatter_dist_psf_mag_perfilter.png}
    \caption{The PSF magnitude residual for the found fakes as function of their matching distance (in [arcsec]).}
    \label{fig:photometric_recovery_vs_astrometric_dist}
\end{figure}

\begin{figure}
    \centering
    \includegraphics[width=0.95\linewidth]{dia/figures/lost_sources_detector_map.png}
    \caption{The scatter map of the lost sources in the detector coordinates. In gray scale in the colorbar we show the object magnitudes.}
    \label{fig:lost_sources_xy}
\end{figure}


\begin{figure}
    \centering
    \includegraphics[width=0.95\linewidth]{dia/figures/flux_pulls.png}
    \caption{The flux pull distribution for all the found fakes, in each filter bandpass. A zero mean, unit dispersion Gaussian distribution function is also included for reference.}
    \label{fig:fake_sources_pulls}
\end{figure}


\begin{figure}
    \centering
    \includegraphics[width=0.95\linewidth]{dia/figures/Efficiency_vs_forced_base_PsfFlux_instFlux_SNR.png}
    \caption{The detection efficiency as function of the PSF estimated S/N ratio of the fake sources. The SNR 1/2 parameter is also included in dashed lines, and represents the 50\% efficiency S/N threshold value (lower is better).}
    \label{fig:eff_vs_snr_fakes}
\end{figure}





