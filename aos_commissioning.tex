\section{Active Optics System Commissioning}
\label{sec:aos_commissioning}

The commissioning of the Active Optics System (AOS) has marked a significant milestone for the Rubin Observatory, 
demonstrating the system's readiness to support high-quality imaging. The first images captured with ComCam 
achieved an impressive 1.7 arcsecond full-width half-maximum (FWHM) image quality, exceptional work of the engineering 
and optical teams during assembly and to the optimization of the Look-Up Table (LUT) using laser tracker data at
the beginning of 2024. Through rigorous testing, frustration and lots of learning, 
we have achieved a system capable of delivering sub-arcsecond image quality with good seeing, 
laying the foundation for the transition to LSSTCam commissioning and the start of survey operations.

\subsection{Initial Alignment}
The initial alignment of the AOS utilized an updated laser tracker nominal frame, ensuring the system was 
consistently brought into focus. These refinements simplified the alignment process, demonstrating the 
value of accurate laser tracker data and paving the way for seamless operations.

\subsection{Coordinate Systems}
Throughout commissioning, significant effort was devoted to understanding and refining coordinate systems. 
We identified a rotation discrepancy in ComCam's installation compared to the expected design, requiring 
adjustments in our alignment procedures. This experience was useful to prepare for similar challenges 
with LSSTCam, equipping the AOS team with strategies to handle future discrepancies.

\subsection{Wavefront estimation}
The wavefront estimator proved robust across diverse observing conditions. 
On dense fields such as 47 Tuc, the estimator provided accurate results for 
all sensors except the central one. Validation against Batoid simulations confirmed 
the good accuracy of Rubin's ray-tracing software.

Advancements included the implementation of sparse Zernikes, which allowed selective 
inclusion of terms while minimizing cross-correlations of modes with identical azimuthal 
dependencies. The forward-model approach (Danish) demonstrated resilience to poor seeing 
and significant defocus, although it remains slower than the TIE method. Improving the 
computational efficiency of both techniques remains a priority for the team.

\subsection{Closed Loop}
Following resolution of initial issues with the AOS pipelines, 
closed-loop operations were achieved across varying elevations. 
Most optical modes were utilized, excluding the three highest-order modes on M2. 
Consistency in results across nights confirmed the need for further refinement of the LUT.

In favorable seeing conditions, the system achieved sub-arcsecond image 
quality, with FWHM as low as 0.65 arcseconds. Zernike-based AOS FWHM 
contributions confirmed the system's convergence to exceptional image 
quality. Autonomous closed-loop operations were run by observers, 
demonstrating the maturity of the system. Preparations are underway 
for a fully autonomous survey-mode triplet-taking block before the 
conclusion of ComCam's on-sky operations.

\subsection{LUT}
The LUT underwent initial validation across elevations, azimuths, and rotator angles, 
leading to incremental improvements. While these updates enhanced performance, 
further refinements are needed to address second-order dependencies. Insights 
from ComCam data will inform these efforts, ensuring readiness for LSSTCam, which 
may present distinct challenges due to its larger focal plane and optical system.

\subsection{Lessons Learned and Next Steps}
\textbf{Lessons Learned}
- Coordinate Systems: Precision and methodical testing of coordinate systems are essential. 
Starting with foundational tests and incrementally increasing complexity ensures reliability.
- Observer Training: Comprehensive documentation, including a subsystem overview and closed-loop procedures, 
significantly enhances observer support capabilities.
- Engagement and Morale: Fun and engaging night summaries boost team morale, fostering a collaborative 
and motivated work environment.
- Transferability: Some ComCam learnings, particularly LUT and coordinate system adjustments, 
will not fully transfer to LSSTCam, requiring repeated validation.

\textbf{Next Steps}
- Conduct step-by-step closed-loop validations for LSSTCam, validating signs and rotations for intentional perturbations.
- Collaborate with the Camera Team to anticipate and mitigate known camera tilts.
- Implement and validate tests tailored to LSSTCam's larger focal plane dimensions.
- Prepare RubinTV and donutViz for full-array LSSTCam mode and automate its execution for all triplet-taking sequences.
- Adapt MTAOS to run as a continuous background task, supporting survey-mode operations.
- Optimize the AOS pipeline for speed, including binning and ISR performance improvements.


