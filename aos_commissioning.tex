\section{Active Optics System Commissioning}
\label{sec:aos_commissioning}

The commissioning of the Active Optics System (AOS) has marked a significant milestone for the Rubin Observatory, 
demonstrating the system's readiness to support high-quality imaging. The first images captured with ComCam 
achieved an impressive 1.7 arcsecond full-width half-maximum (FWHM) image quality, a testament to the exceptional metrology
work of the engineering and optical teams during assembly and to the optimization of the Look-Up Table (LUT) for all the active optics components using 
laser tracker data as well as mirror force balance data throughout 2024. Through rigorous testing, frustration and lots of learning, 
we have achieved a system capable of delivering sub-arcsecond image quality in good seeing, 
laying the foundation for the transition to LSSTCam commissioning and the start of survey operations.

\subsection{Initial Alignment}
The initial alignment of the AOS utilized an updated laser tracker nominal frame issued from our Final Element Model, ensuring the system was 
consistently brought into focus. Combined with a thorough measurement of the impact of gravity on the telescope, 
these refinements simplified the alignment process, demonstrating the 
value of accurate laser tracker data and paving the way for seamless operations.
Once we were able to get on-sky images using curvature wavefront sensing, we finalized the optimal hexapods position that would ensure a well aligned system at the start of the night. 

\subsection{Coordinate Systems}
During the first few weeks, significant effort was devoted to understanding and refining coordinate systems at different steps of the Active optics closed-loop process (wavefront sensor estimation and correction calculation). We conducted the test by introducing one degree of freedom at a time and correcting for it. 
We identified a rotation discrepancy in ComCam's installation compared to the expected design, requiring 
adjustments in our alignment procedures. This experience was useful to prepare for similar challenges 
with LSSTCam, equipping the AOS team with strategies to handle future discrepancies. 

\subsection{Wavefront estimation}
The wavefront estimator proved robust across diverse observing conditions of seeing, mount elevation and a few filters (r,i and y band) 
On dense fields such as 47 Tuc or NGC 253, the estimator provided accurate results for 
all sensors except the central one.  Comparison of observed PSFs with simulations 
confirmed the good accuracy of Rubin's ray-tracing software, \texttt{Batoid}.

Wavefront estimation and closed-loop convergence has been demonstrated using two different
wavefront estimation algorithms: the \texttt{TIE}, which is an inversion method, and \texttt{Danish}, which
is a forward modeling method. Other advancements include the implementation of sparse Zernikes, 
allowing selective inclusion of Zernike polynomial terms while minimizing cross-contamination 
of modes with identical azimuthal dependencies.

Despite delivering good optical quality, Zernike measurements indicate persistently high
levels of defocus and some amount of astigmatism. We are continuing to investigate the
source and impact of these measurements.

\subsection{Closed Loop}
Following resolution of initial issues with the AOS pipelines, 
closed-loop operations were achieved across varying elevations. 
Most optical modes were utilized, excluding the three highest-order modes on M2. 
Consistency in results across nights confirmed the need for further refinement of the LUT.

In favorable seeing conditions, the system achieved sub-arcsecond image 
quality, with FWHM as low as 0.65 arcseconds, with a variation across the field less than 0.1 arcsecond (TBC). Zernike-based AOS FWHM 
contributions confirmed the system's convergence to exceptional image 
quality. Autonomous closed-loop operations were run by observers, 
demonstrating the maturity of the system. Preparations are underway 
for a fully autonomous survey-mode triplet-taking block before the 
conclusion of ComCam's on-sky operations.

The process still takes 5min though, requiring 5 or more iterations. Tuning the closed-loop gain and making further adjustments to improve computational 
efficiency remains a priority for the team.

\subsection{LUT}
The LUT underwent initial validation across elevations, azimuths, and rotator angles, 
leading to incremental improvements. While these updates enhanced performance, 
further refinements are needed to address second-order dependencies. Insights 
from ComCam data will inform these efforts, ensuring readiness for LSSTCam, which 
may present distinct challenges due to its larger focal plane and optical system.

\subsection{Lessons Learned and Next Steps}
\textbf{Lessons Learned}
- Coordinate Systems: Precision and methodical testing of coordinate systems are essential. 
Starting with foundational tests and incrementally increasing complexity ensures reliability.
- Observer Training: Comprehensive documentation, including a subsystem overview and closed-loop procedures, 
significantly enhances observer support capabilities.
- Engagement and Morale: Fun and engaging night summaries boost team morale, fostering a collaborative 
and motivated work environment.
- Transferability: Some ComCam learnings, particularly LUT and coordinate system adjustments, 
will not fully transfer to LSSTCam, requiring repeated validation.

\textbf{Next Steps}
- Conduct step-by-step closed-loop validations for LSSTCam, validating signs and rotations for intentional perturbations.
- Collaborate with the Camera Team to anticipate and mitigate known camera tilts.
- Implement and validate tests tailored to LSSTCam's larger focal plane dimensions.
- Prepare RubinTV and donutViz for full-array LSSTCam mode and automate its execution for all triplet-taking sequences.
- Adapt MTAOS to run as a continuous background task, supporting survey-mode operations.
- Optimize the AOS pipeline for speed, including binning and ISR performance improvements.


