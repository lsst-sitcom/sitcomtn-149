\section{Image Inspection}
\label{sec:image_inspection}

The prospects for human-based inspection of the vast number of images to be
produced by the Rubin/LSST are unavoidably going to be limited to a very small
fraction of the dataset produced (even nightly, let alone for the full 10-year
survey).  Yet, the potential value of getting human eyes on the images (including
the raw, minimally processed, and final processed and calibrted stages), is
immense, in particular for identifying patterns that are easily spotted by eye
yet tend to evade most modern automated image quality assessment protocols.

Every dataset from any given observation program comes with its own unique set
of ``features'' stemming from the observatory structres, optics, camera,
detectors, electronics, observation strategy, the night sky, etc.  This makes
eyeball inspection particulalry valuable in the early days of commissioning.
Fortuitously, the amount of energy and enthusiasm from internal project members
are at extrememly high levels at this stage, so there is no shortage of voluntary
effort for human visial inpsection for the commissioning phase of the LSSTComCam.
This effort do date has largely proceded via an informal
see-something-say-something scheme, with many users posting their latest findings
on the internal staff Slack channels (namely the \#sciunit-image=inspection channel,
but also prominently in other channels, \#sciunit-lsb, \#validation-team,
\#embargo-beautiful-images, \#ops-satellites, to name a few).

Anomolies and peculiarities reported to date along with details of and/or pointers
to further study and explanation (where applicable) include:
\begin{itemize}

\item \textbf{ghosts}: largely seen around bright stars (including those falling
  outside of the FOV) and attributed to multi-bounce reflections off of the focal
  plane and the LSSTComCam optics.  The presence and appearance of the ghosts
  are very well predicted by Josh Meyers' ``batoid ray tracer'', even in its
  current unoptimized state.  It is noted that the nature of the ghosts will be
  quite different for the full LSSTCam, but this initial level of understanding
  of this feature is promissing and useful for honing characterization and
  mitigation strategies.

\item \textbf{stray/scattered light}: prominent ring-shaped waffle/corduroy-like
  features seen early on were identified as originating from a blinking light on a
  crane that was left on. Large irregular ring-shaped ``spots'' were attributed to
  the laser tracker (an issue in access to turn it off was noted).

\item \textbf{streaks}: see Section~\ref{sec:dia_transient_variable}

\item \textbf{kettlebell \& trefoil-shaped PSFs}: see Section~\ref{sec:aos_commissioning}.
  This issue was further investigated in the context of the PSF modelling itself on such
  irregularly shaped PSFs.  It was noted that the preferred ``piff'' algorigthm (which
  is used for the final characterization model) was performing worse that the ``psfex''
  algorithm (which, due to its relative speed, is used in the initial characterization).
  As such, we temporarily switched to using only ``psfex'' until the image quality was
  improved.  Given the incredible and expeditious work of the AOS team, we are already
  confident that we can switch back to our preferred configurartion.  (This has also
  spurred a study into the reason why ``piff'' seems to struggle with these shapes and
  will hopefully lead to improvements there, see
  Section~\ref{sec:delivered_image_quality_and_psf}).

\item \textbf{background subtraction issues}: see Section~\ref{sec:low_surface_brightness}

\item \textbf{repeated patchwork gradients}: I admittedly have yet to read the 114+ reply
  thread for this one, but see issues like crosstalk and electronics readouts, etc., have
  been speculation, so this is most likely to be further assessed in
  Section~\ref{sec:isr}-related work.

\item \textbf{trailed bright sources}: due to tracking errors.  These are particularly
  difficult to identify with automated image quality metrics (there are some ideas
  about using AI to identify tracking error-based image degradation floating around,
  but we also hope to be able to confidently rely on data from the EDF to indicate
  tracking issues).

\end{itemize}

\subsection{Future Endeavors}

A major goal for future image inspction efforts is to develop a more systematic
system for identifying and reporting issues and, where possible, alerting the
relevant stakeholders for further investigations into mitigating the problem.

The most promissing venue currently under development is the deployment of an
``Exposure Checker'' using the same infrastructure used for the Dark Energy Survey
(see https://arxiv.org/pdf/1511.03391 and demo at
http://des-exp-checker.pmelchior.net/index.html).
**** ALEX TO ADD SOME WORDS??? ***

Also recently implemented for visual inspection is a rendering of the 3-color coadd
HIPS maps produced regularly during the nightly validation and DRP processing runs.
Such images are invaluable for highlighting myriad issues at the coadd level (often
indicating a need to drill-down to the visit-level for a full diagnosis, while
providing significant clues on where to look first).
