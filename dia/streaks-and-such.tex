\subsection{Satellite Streaks}

As orbital space becomes increasingly crowded, we expect to see many bright streaks, flares, and glints due to
satellites and other reflective human-made objects orbiting the Earth, with the majority of the population is in low-Earth orbit (LEO).

As expected, many ComCam detector-visit images clearly show streaks. Visual inspection of nearly all ComCam images to date are being recorded on a best-effort basis in a Confluence Database dubbed ``ComCam Satellite Spotter,'' and as of 2024 Nov 25 there are over 500 rows. 
\begin{itemize}
\item Straight bright linear feature, typically at least 20 pixels wide, that crosses one or more detectors and goes off the edge (typical of most LEO satellites, such as Starlink)
\item Shorter version of the above, with clear start and/or endpoints, which usually indicates the object imaged is located at a higher-than-LEO orbital altitude (and/or the exposure integration time was unusually short)
\item Intermittent linear feature, i.e., a dashed line, due to different parts of the satellite having different reflective properties
\item A flare or glint brightening event that fades in and out along a linear trajectory, either isolated or as part of a longer streak
\item Actually a bright star diffraction spike
\item Actually a cosmic ray that was not repaired
\item Variation of any of the above but in out-of-focus donut form (interestingly, depending on altitude, certain streaks may appear either in- or out-of-focus when stars appear as donuts)
\end{itemize}

Thanks to ComCam's relatively small field of view and the satellite population being as small as it ever will
be during Rubin Commissioning and Operations, we have not yet seen an overwhelmingly bright satellite (e.g.,
BlueWalker 3 or one of the BlueBirds, all operated by AST SpaceMobile). Only a couple instances have streaks
bright enough to induce visually-obvious crosstalk ``secondary streaks;'' the majority of streaks are
relatively faint and the only portion of the image impacted are regions overlapping with the streak itself.
Reliably determining streak width is an ongoing challenge, as they are wider than the PSF, and some of
the brighter streaks have noticeably extended stray light wings.
