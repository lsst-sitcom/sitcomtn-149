\section{System Performance Analysis}
\label{sec:system_performance_analysis}

\newcommand{\TMAMotionSettings}{\href{https://rubinobs.atlassian.net/wiki/spaces/LSSTCOM/pages/53741249/TMA+Motion+Settings}{TMA Motion Settings}}
\newcommand{\testCase}[1]{\href{https://rubinobs.atlassian.net/projects/BLOCK?selectedItem=com.atlassian.plugins.atlassian-connect-plugin:com.kanoah.test-manager__main-project-page#!/v2/testCase/#1}{#1}}

Topics to convert into text

\begin{itemize}
    \item M1M3 and M2 glass installed on the Simonyi Survey Telescope.
    \item Since then, we have been operating the telescope with limited velocity,
    acceleration, and jerk limits following the performances defined in \TMAMotionSettings.
    \item For each configuration, defined in terms of a percentage of the maximum
    velocity, acceleration, and jerk, we ran multiple gateway tests.
    \item The gateway tests are described in the subsection \ref{subsec:gateway_tests} below.
\end{itemize}

\subsection{Gateway Tests}
\label{subsec:gateway_tests}

We started the ComCam on Sky test campaign using Simonyi Telescope with limited
performance, described as a percentage of the maximum velocity, acceleration,
and jerk limits. The performance is defined in \TMAMotionSettings Concluence page.

Before we can increase the telescope performance, we need to perform a set of
tests that ensure that the system will respond safely to the new velocity, acceleration,
and jerk. These tests are called gateway tests. Here is the list of all the tests.

\begin{itemize}
    \item \testCase{BLOCK-T227} Dynamic Tests at El = 34º short and long slews
    \item \testCase{BLOCK-T294} Dynamic Tests at El = 70º short and long slews
    \item \testCase{BLOCK-T231} TMA Azimuth Brake Test
    \item \testCase{BLOCK-T240} TMA Elevation Brake Distance
    \item \testCase{BLOCK-T241} M2 closed-loop break-out brake test during TMA slew
\end{itemize}


\subsubsection{Long and short slews at different elevations}
\label{subsubsec:long_and_short_slews}

These tests ensure that the force balance systems on M1M3 and on M2 can protect
the mirrors on different telescope positions and while slewing. As we increase
velocity, acceleration, and jerk limits, both mirrors suffer higher inertial
forces and the force actuators must counteract them.

%% TODO @b1quint - Ask Pablo, Holger, and Gabriele about the criteria for the
%% force balance systems for M2.
For M1M3, the criteria is to keep the measured forces on the hardpoint actuators
below the operational limit (15\% the breakaway limit). For M2, the criteria is
??????? (check with Holger, Gabriele, and Pablo).

% Detailed analysis (both TNs are still under development):
% \begin{itemize}
%     \item \href{https://sitcomtn-092.lsst.io/}{SITCOM-TN092 M1M3 Force Balance System \item Inertia Compensation}
%     \item \href{https://sitcomtn-147.lsst.io/}{SITCOM-TN147 M2 Response to short and long slews}
% \end{itemize}



\begin{figure}
    \centering
    \includegraphics[width=0.8\textwidth]{spa/10_vel_acc_jerk/M1M3_short_long_slews_10_histogram.png}
    \caption{Number of slews with minimum/maximum measured forces on the M1M3 hardpoint actuators.}
    \label{fig:m1m3_short_long_slews}
    \end{figure}

\begin{figure}
    \centering
    \includegraphics[width=0.8\textwidth]{spa/M2_short_long_slews_axial_measured_force_10.png}
    \caption{Measured axial force on the M2 force actuators during short and long slews.}
    \label{fig:m2_short_long_slews_axial}
    \end{figure}

\begin{figure}
    \centering
    \includegraphics[width=0.8\textwidth]{spa/M2_short_long_slews_Tangent_measured_forces_TMA_10.png}
    \caption{Measured tangent force on the M2 force actuators during short and long slews.}
    \label{fig:m2_short_long_slews_tangent}
    \end{figure}

\begin{figure}
    \centering
    \includegraphics[width=0.8\textwidth]{spa/M2_short_long_slews_tangent_force_errors_10.png}
    \caption{Measured tangent force errors on the M2 force actuators during short and long slews.}
    \label{fig:m2_short_long_slews_tangent_errors}
    \end{figure}


\subsubsection{M2 close-loop breakout tests}
\label{subsubsec:m2_close_loop_breakout_tests}

\begin{itemize}
    \item \href{https://rubinobs.atlassian.net/projects/LVV?selectedItem=com.atlassian.plugins.atlassian-connect-plugin:com.kanoah.test-manager__main-project-page#!/v2/testCase/LVV-T3034}{LVV-T3034 M2 closed-loop break-out during TMA slew}
    \item \href{https://rubinobs.atlassian.net/projects/BLOCK?selectedItem=com.atlassian.plugins.atlassian-connect-plugin:com.kanoah.test-manager__main-project-page#!/v2/testCase/BLOCK-T241}{BLOCK-T241 M2 closed-loop break-out brake test during TMA slew}
\end{itemize}

\begin{figure}
    \centering
    \includegraphics[width=0.8\textwidth]{spa/M2_cl_breakout_10_axial_force.png}
    \caption{M2 axial forces during the closed-loop breakout test.}
    \label{fig:m2_closed_loop_breakout_axial_force}
    \end{figure}

\begin{figure}
    \centering
    \includegraphics[width=0.8\textwidth]{spa/M2_cl_breakout_10_tangent_forces.png}
    \caption{M2 tangent forces during the closed-loop breakout test.}
    \label{fig:m2_closed_loop_breakout_tangent_force}
    \end{figure}

\begin{figure}
    \centering
    \includegraphics[width=0.8\textwidth]{spa/M2_cl_breakout_10_tangent_force_errors.png}
    \caption{M2 tangent force errors during the closed-loop breakout test.}
    \label{fig:m2_closed_loop_breakout_tangent_force_errors}
    \end{figure}


\subsubsection{TMA azimuth and elevation brake tests}
\label{subsubsec:tma_azimuth_and_elevation_brake_tests}

Test cases associated:
\begin{itemize}
    \item \href{https://rubinobs.atlassian.net/projects/BLOCK?selectedItem=com.atlassian.plugins.atlassian-connect-plugin:com.kanoah.test-manager__main-project-page#!/v2/testCase/BLOCK-T231}{BLOCK-T231 TMA Azimuth Brake Test}
    \item \href{https://rubinobs.atlassian.net/projects/BLOCK?selectedItem=com.atlassian.plugins.atlassian-connect-plugin:com.kanoah.test-manager__main-project-page#!/v2/testCase/BLOCK-T240}{BLOCK-T240 TMA Elevation Brake Distance}
\end{itemize}

\begin{figure}
    \centering
    \includegraphics[width=0.8\textwidth]{spa/TMA_Az_brake_test_10.png}
    \caption{TMA azimuth brake test.}
    \label{fig:tma_azimuth_brake}
    \end{figure}

\begin{figure}
    \centering
    \includegraphics[width=0.8\textwidth]{spa/TMA_El_brake_test_10.png}
    \caption{TMA elevation brake test.}
    \label{fig:tma_elevation_brake}
    \end{figure}

\subsection{Night Performance}

Statistical reports/summaries during the night?
\begin{itemize}
    \item Measured m1m3 hardpoint histograms min/max HP forces.
    \item FRACAS-158 / SITCOMTN-081 / SITCOM-1758 - Oscillations on HP forces and on azimuth torques
\end{itemize}
