\documentclass[SE,lsstdraft,authoryear,toc]{lsstdoc}
\input{meta}

% Package imports go here.

% Local commands go here.

%If you want glossaries
%\input{aglossary.tex}
%\makeglossaries

\title{An Interim Report on the ComCam On-Sky Campaign}

% This can write metadata into the PDF.
% Update keywords and author information as necessary.
\hypersetup{
    pdftitle={An Interim Report on the ComCam On-Sky Campaign},
    pdfauthor={Robert Lupton},
    pdfkeywords={}
}

% Optional subtitle
% \setDocSubtitle{A subtitle}

\author{%
Robert Lupton
}

\setDocRef{SITCOMTN-149}
\setDocUpstreamLocation{\url{https://github.com/lsst-sitcom/sitcomtn-149}}

\date{\vcsDate}

% Optional: name of the document's curator
% \setDocCurator{The Curator of this Document}

\setDocAbstract{%
A summary of what we have learned from the initial period of ComCam observing
}

% Change history defined here.
% Order: oldest first.
% Fields: VERSION, DATE, DESCRIPTION, OWNER NAME.
% See LPM-51 for version number policy.
\setDocChangeRecord{%
  \addtohist{1}{YYYY-MM-DD}{Unreleased.}{Robert Lupton}
}


\begin{document}

% Create the title page.
\maketitle
% Frequently for a technote we do not want a title page  uncomment this to remove the title page and changelog.
% use \mkshorttitle to remove the extra pages

% ADD CONTENT HERE
% You can also use the \input command to include several content files.

\appendix
% Include all the relevant bib files.
% https://lsst-texmf.lsst.io/lsstdoc.html#bibliographies
\section{References} \label{sec:bib}
\renewcommand{\refname}{} % Suppress default Bibliography section
\bibliography{local,lsst,lsst-dm,refs_ads,refs,books}

% Make sure lsst-texmf/bin/generateAcronyms.py is in your path
\section{Acronyms} \label{sec:acronyms}
\addtocounter{table}{-1}
\begin{longtable}{p{0.145\textwidth}p{0.8\textwidth}}\hline
\textbf{Acronym} & \textbf{Description}  \\\hline

2D & Two-dimensional \\\hline
3D & Three-dimensional \\\hline
ADU & Analogue-to-Digital Unit \\\hline
AI & Artificial Intelligence \\\hline
AOS & Active Optics System \\\hline
CBP & Collimated Beam Projector \\\hline
CCD & Charge-Coupled Device \\\hline
CNN & Convolutional Neural Network \\\hline
COSMOS & Cosmic Evolution Survey \\\hline
DC2 & Data Challenge 2 (DESC) \\\hline
DE & dark energy \\\hline
DECaLS & The Dark Energy Camera Legacy Survey \\\hline
DECam & Dark Energy Camera \\\hline
DES & Dark Energy Survey \\\hline
DESC & Dark Energy Science Collaboration \\\hline
DIA & Difference Image Analysis \\\hline
DIMM & Differential Image Motion Monitor \\\hline
DM & Data Management \\\hline
DR10 & Data Release 10 \\\hline
DR2 & Data Release 2 \\\hline
DRP & Data Release Production \\\hline
EDFS & Euclid Deep Field South \\\hline
FGCM & Forward Global Calibration Model \\\hline
FOV & field of view \\\hline
FRACAS & Failure Reporting Analysis and Corrective Action System \\\hline
FWHM & Full Width at Half-Maximum \\\hline
GBDES & Gary Bernstein Dark Energy Survey \\\hline
HIPS & Hierarchical Progressive Survey (IVOA standard) \\\hline
HSC & Hyper Suprime-Cam \\\hline
HST & Hubble Space Telescope \\\hline
ISR & Instrument Signal Removal \\\hline
ITL & Imaging Technology Laboratory (UA) \\\hline
JPL & Jet Propulsion Laboratory (DE ephemerides) \\\hline
LATISS & LSST Atmospheric Transmission Imager and Slitless Spectrograph \\\hline
LSB & Low Surface Brightness \\\hline
LSST & Legacy Survey of Space and Time (formerly Large Synoptic Survey Telescope) \\\hline
LUT & Look-Up Table \\\hline
LVV & LSST Verification and Validation \\\hline
M1M3 & Primary Mirror Tertiary Mirror \\\hline
M2 & Secondary Mirror \\\hline
ML & Machine Learning \\\hline
MODTRAN & MODerate resolution TRANsmission model \\\hline
NGC & New General Catalogue \\\hline
PSF & Point Spread Function \\\hline
QA & Quality Assurance \\\hline
QE & quantum efficiency \\\hline
RA & Risk Assessment \\\hline
RMS & Root-Mean-Square \\\hline
RTN & Rubin Technical Note \\\hline
SDSS & Sloan Digital Sky Survey \\\hline
SE & System Engineering \\\hline
SED & Spectral Energy Distribution \\\hline
SITCOM & System Integration, Test and Commissioning \\\hline
SLAC & SLAC National Accelerator Laboratory \\\hline
SOAR & Southern Astrophysical Research Telescope \\\hline
SSI & Synthetic Source Injection \\\hline
SSP & Solar System Processing \\\hline
TBC & To Be Confirmed \\\hline
TMA & Telescope Mount Assembly \\\hline
UA & University of Arizona \\\hline
ZTF & Zwicky Transient Facility \\\hline
\end{longtable}

% If you want glossary uncomment below -- comment out the two lines above
%\printglossaries





\end{document}
