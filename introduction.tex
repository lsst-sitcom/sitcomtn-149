\section{Introduction}
\label{sec:introduction}

The Vera C. Rubin Observatory on-sky commissioning campaign using the Commissioning Camera (ComCam) began on 24 October 2024 and is forecasted to continue through mid-December 2024.
This interim report provides a concise summary of our understanding of the integrated system performance based tests and analyses conducted during the first weeks of the ComCam on-sky campaign.
The emphasis is distilling and communicating what we have learned about the system.
The report is organized into sections to describe major activities during the campaign, as well as multiple aspects of the demonstrated system and science performance.

\subsection{Charge}

We identify the following high-level goals for the interim report:

\begin{itemize}

    \item \textbf{Rehearse workflows for collaboratively developing documentation} to describe our current understanding of the integrated system performance, e.g., to support the development of planned Construction Papers and release documentation to support the Early Science Program \citedsp{RTN-011}.
    This report represents an opportunity to collectively exercise the practical aspects of developing documentation in compliance with the policies and guidelines for information sharing during commissioning \citedsp{sitcomtn-076}.

    \item \textbf{Synthesize the new knowledge} gained from the ComCam on-sky commissioning campaign to inform the optimization of activities between the conclusion of the ComCam campaign and the start of the on-sky campaign with the LSST Camera (LSSTCam).

    \item \textbf{Inform the Rubin Science Community} on the progress of the on-sky commissioning campaign using ComCam.

\end{itemize}

Other planned systems engineering activities will specifically address system-level verification (\citedsp{lse-29} and \citedsp{lse-30}) using tests and analysis from the ComCam campaign.
While the analyses in this report will likely overlap with the generation of verification artifacts for systems engineering, and system-level requirement specifications will serve as key performance benchmarks for interpreting the progress to date, formal acceptance testing is not an explicit goal of this report.

\textbf{The groups within the Rubin Observatory project working on each of the activities and performance analyses are charged with contributing to the relevant sections of the report.}
The anticipated level of detail for the sections ranges from a paragraph up to a page or two of text, depending on the current state of understanding, with quantitative performance expressed as summary statistics, tables, and/or figures.
The sections refer to additional supporting documentation, e.g., analysis notebooks, other technotes with further detail, as needed.
Given the timelines for commissioning various aspects of the system, it is natural that some sections will have more detail than others.

The anticipated milestones for developing this interim report are as follows:

\begin{itemize}

    \item 18 Nov 2024: Define charge

    \item 4 Dec 2024: First drafts of report sections made available for internal review

    \item 11 Dec 2024: Revised drafts of report sections made available for internal review; editing for consistency and coherency throughout the report

    \item 18 Dec 2024: Initial version of report is released

\end{itemize}

\begin{warning}[On-sky Pixel Image Embargo]
    All pixel images and representations of pixel images of any size field of view, including individual visit images, coadd images, and difference images based on ComCam commissioning on-sky observations must be kept internal to the Rubin Observatory Project team, and in particular, cannot be included in this report.
    Embargoed pixel images can only be referenced as authenticated links.
    See \citedsp{sitcomtn-076} for details.
\end{warning}