% We’re writing a tech note (SITCOMTN-149) to capture our
% understanding of the state of the system during on-sky commissioning
% campaign with ComCam.  Please use this ticket to capture the work.
%
% From the introduction:
%
% The Vera C. Rubin Observatory on-sky commissioning campaign using
% the Commissioning Camera (ComCam) began on 24 October 2024 and is
% forecasted to continue through mid-December 2024. This interim
% report provides a concise summary of our understanding of the
% integrated system performance based tests and analyses conducted
% during the first weeks of the ComCam on-sky campaign. The emphasis
% is distilling and communicating what we have learned about the
% system. The report is organized into sections to describe major
% activities during the campaign, as well as multiple aspects of the
% demonstrated system and science performance.
%
% Charge:
%
% The groups within the Rubin Observatory project working on each of
% the activities and performance analyses are charged with
% contributing to the relevant sections of the report. The anticipated
% level of detail for the sections ranges from a paragraph up to a
% page or two of text, depending on the current state of
% understanding, with quantitative performance expressed as summary
% statistics, tables, and/or figures.  The objective for this document
% is to summarize the state of knowledge of the system, rather than
% how we got there or “lessons learned”. The sections refer to
% additional supporting documentation, e.g., analysis notebooks, other
% technotes with further detail, as needed. Given the timelines for
% commissioning various aspects of the system, it is natural that some
% sections will have more detail than others.

\section{Instrument Signature Removal}
\label{sec:isr}
\newcommand{\czw}[1]{
  \textbf{CZW: }\textcolor{red}{#1}
}
The quality of the instrument signal removal (ISR) has improved during commissioning, as we create and deploy updated calibration products that better represent the LSSTComCam system.
The following discussion summarizes our current understanding of a variety of features, both expected and newly seen on LSSTComCam, and presents our expected prognosis of the behavior of the full LSSTCam.,

\subsection{Phosphorescence}

There are regions on some of the detectors (most visible in \czw{R22_S01}) which show bright emission, particularly at bluer wavelengths, as shown in Figure \ref{fig:isr_phosphorescence_example}
This is believed to be caused by a thin layer of remnant photo-resist from the manufacturing process that remained on the detector surface, and is now permanent due to the subsequent addition of the anti-reflective coating.
This material is known to be phosphorescent, explaining why these regions show more strongly in the blue.

The initial studies of this (Figure \ref{fig:isr_phosphorescence_decay} show that these features can continue to emit light up to \czw{N seconds} after they've been illuminated.
Due to the long duration of these features, we decided to place manual defect masks over the worst regions, and 

\begin{figure}
  \caption{The phosphorescence seen in \czw{R22_S01}, shown here in a DARK exposure taken after a series of twilight flats (exposure=2024112000065).  This material absorbs light at bluer wavelengths and re-emit that energy over a wide range of wavelengths, and for \czw{add some characterization of the time scales}.}
  \ref{fig:isr_phosphorescence_example}
\end{figure}

\begin{figure}
  \caption{Brightness of a phosphorescent area (defined as \czw{XYZ}) as a function of time since the last illumination, in a series of \czw{bias?} exposures.  Measured by \czw{I remember seeing this, but don't remember who}.}
  \ref{fig:isr_phosphorescence_decay}
\end{figure}

Expected loss ~3.5\%, consistent with estimates of ``less than an amplifier.'' 
The LSSTCam ITL detectors are believed to have been cleaned better, but this may still be an issue.
  
\subsection{Vampire pixels}

Bright pixel with axisymmetric ``depressed'' region.
Appear to conserve charge.
More common than thought.  We're masking with 200\%-of-flat threshold, with X pixel radius circle masks
Two detectors in the full camera may have this at the level of R22_S10.
All ITL have a few.

\begin{figure}
  \caption{A close up of one of the largest vampire pixels.}
\end{figure}

\begin{figure}
  \caption{A view of detector \czw{R22_S10}, which has a large number of less significant vampire pixels.}
\end{figure}


\subsection{Saturated star effects}

Saturated stars end up with dark trails that stretch to the top and bottom of the detector (across channel stops)
Masking will be the solution here as well.  Saturation threshold + width configurations.

\begin{figure}
  \caption{Example of a bright star with full-detector bleed.}
\end{figure}

\subsection{Gain ratios}
We have different observed gain ratios between the lab data and the on-sky data.
Two lab scans come up with different results, and on sky is different still.
Does not seem to correlate with REB temperature, time, VBB etc.
It does appear to be stable, as we've only needed to apply the correction once.

\begin{figure}
  \caption{Example of gain ratio mismatch.}
\end{figure}

\begin{figure}
  \caption{measured ratios}
\end{figure}

\subsection{Crosstalk}

Crosstalk is treated reasonably well by the current set of coefficients: averaged ITL measurements taken on LSSTCam.
Expect those measurements (specifically the filtered version to exclude outliers) will be sufficient at the start of LSSTCam comissioning.
\czw{I want to have numbers and figures here.}

\subsection{Twilight flats}

Twilight flats have worked better than expected.
We now have at least an initial set of flats for ugrizy.
Star print through is one problem we've seen, but this will improve as we take more data + rotations + etc.
Vampire pixels do show up

