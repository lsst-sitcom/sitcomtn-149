\section{Executive Summary}
\label{sec:summary}

The Vera C. Rubin Observatory project team has completed a first series of on-sky engineering tests demonstrating the end-to-end functionality of the Simonyi Survey Telescope's hardware and software systems using an engineering test camera, the Commissioning Camera (ComCam).

\subsection{Commissioning Camera (ComCam)}

ComCam mass distribution, physical envelope, and interfaces designed to match LSSTCam.
ComCam focal plane has single raft w/ 3x3 mosaic of (4K $\times$ 4K) ITL science sensors (144 Mpix total).
For comparison, LSSTCam comprises 21 rafts w/ both ITL and e2v sensors (total of 189 science sensors; 3.2 Gpix) + 4 dedicated corner rafts for guiding and wavefront estimation.
Piston entire ComCam focal plane to acquire intra- and extra-focal images of the pupil (``donuts'') for AOS commissioning.
Same plate scale as LSSTCam (0.2 arcsec / pixel).
ComCam filter exchanger holds 3 physical filters at a time.
ComCam Refrigeration Pathfinder allowed commissioning of the refrigeration system prior to arrival of LSSTCam.

\subsection{Accomplishments}

The primary goals of the campaign were to use ComCam to

\begin{enumerate}
    \item learn how to more efficiently commission the LSST Camera (LSSTCam) on sky
    \item optically align the telescope and verify capability to deliver acceptable image quality for the smaller ComCam field of view
\end{enumerate}

The ComCam on-sky campaign accomplished the primary goals, and much more

\begin{itemize}
    \item Telescope delivers crisp images
    \item Validated and refined AOS open loop system + demonstrated AOS closed loop system
    \begin{itemize}
        \item Telescope focused and aligned at start of each night using laser tracker
        \item Using almost all degrees of freedom
        \item multiple wavefront estimation algorithms
        \item tested range of stellar densities
        \item AOS in LSST Wide-Fast-Deep survey emulation mode
    \end{itemize}
    \item Scheduler driving on-sky operations, including with Feature Based Scheduler
    \item System can acquire visits efficiently at a given target field
    \item ComCam instrument reliable and effective
    \item Telescope tracking and boresight stability have been superb
    \item System optical throughput consistent with a priori expectations
    \item Data transfer and downstream image analysis software working well, including
    \begin{itemize}
        \item Quicklook campaign at summit supporting nighttime operations
        \item Prompt processing and data release processing campaigns at USDF; running at USDF within a few minutes; Generating (but not publishing) alerts; Finding asteroids; Real-bogus classifier running in alert production pipeline; Minimal impact due to cosmic rays
    \end{itemize}
    \item Astrometric and photometric calibration are on track to meet design requirements
    \begin{itemize}
        \item Stellar astrometric repeatability below 10 mas using only single-frame calibration (i.e., before running global solution w/ gbdes algorithm)
        \item Photometric repeatability for bright stars at level consistent with the statistical Poisson noise photometric errors, with contribution from calibration errors at few mmag level
    \end{itemize}
    \item Collimated Beam Projector can co-point w/ telescope and acquire wavelength scans
    \item Demonstrated stuttered imaging and guider mode imaging to evaluate high-frequency contributions to delivered image quality (1-100 Hz)
    \item Exercised end-to-end workflows; new observing staff coming on board; team is strong
\end{itemize}

\subsection{Areas of Investigation and Further Development}

\begin{itemize}
    \item Interpreting stray and scattered light
    \item Contributions to delivered image quality
    \begin{itemize}
        \item Understand discrepancies between AOS residual and delivered image quality
        \item Analysis of stuttered and guider mode imaging
        \item AOS closed loop convergence optimization
        \item M1M3 thermal
        \item Rubin Obs DIMM
        \item Dome seeing monitor
        \item Dome louvers + light and wind screen installation
    \end{itemize}
    \item Calibration harware
    \begin{itemize}
        \item Analysis of CBP data from ComCam; red light leak in g filter
        \item Flat field screen and illumination system installation and commissioning
    \end{itemize}
    \item Overall system reliability and efficiency
    \item Increasing TMA motion
\end{itemize}