\section{Executive Summary}
\label{sec:summary}

The \VeraRubinObservatory project team has completed a first series of on-sky engineering tests demonstrating the end-to-end functionality of the Simonyi Survey Telescope's hardware and software systems using an engineering test camera, the Commissioning Camera (ComCam).

\begin{note}[Versioning Note]
    This interim report provides a preliminary technical overview of the ComCam on-sky campaign based on analyses through early January 2025.
\end{note}

\emph{}

\subsection{Commissioning Camera (ComCam)}

LSSTComCam mass distribution, physical envelope, and interfaces were designed to match LSSTCam \citep{10.71929/rubin/2571927}.
ComCam focal plane has single raft with 3 $\times$ 3 mosaic of (4K $\times$ 4K) ITL science sensors (144 Mpix total).
For comparison, LSSTCam comprises 21 rafts with both ITL and e2v sensors (total of 189 science sensors; 3.2 Gpix) + 4 dedicated corner rafts for guiding and wavefront estimation.
Because ComCam has no dedicated corner rafts, it was necessary to piston the entire ComCam focal plane in
order to acquire intra- and extra-focal images of the pupil (``donuts'') for AOS commissioning.
Same plate scale as LSSTCam (0.2 arcsec / pixel).
The field of view is 40 arcmin $\times$ 40 arcmin.
ComCam filter exchanger holds 3 physical filters at a time.
% ComCam Refrigeration Pathfinder allowed commissioning of the refrigeration system prior to arrival of LSSTCam.

\subsection{Accomplishments}

The primary goals of the campaign were to use ComCam to

\begin{enumerate}
    \item learn how to more efficiently commission the LSST Camera (LSSTCam) on sky
    \item optically align the telescope and verify capability to deliver acceptable image quality for the smaller ComCam field of view
\end{enumerate}

The ComCam on-sky campaign accomplished the primary goals, and much more:

\begin{itemize}
    \item \textbf{The telescope delivers crisp images.}
    The median delivered image quality during the campaign for commanded in-focus images, quantified in terms of the PSF FWHM, was $\sim1.1$ arcseconds.
    The best images have delivered PSF FWHM $\sim0.65$ arcseconds.
    Multiple analysis approaches suggest that Rubin Observatory is already capable of achieving a system contrbution to the delivered image quality of $\sim0.4$ arcseconds (i.e., accounting for the atmosphere seeing contribution), even with several of the Observatory environmental controls not yet fully in place.
    \item \textbf{The team validated and further refined the AOS open loop system, and demonstrated the AOS closed loop system functionality with a variety of configurations and environmental conditions.}
    %\item Validated and refined AOS open loop system + demonstrated AOS closed loop system
    \begin{itemize}
        \item Telescope focused and coarsely aligned at start of each night using laser tracker,
          then fine-tuned using wavefront sensing
        \item Using almost all degrees of freedom
        \item multiple wavefront estimation algorithms
        \item tested range of stellar densities
        \item AOS in LSST Wide-Fast-Deep survey emulation mode
    \end{itemize}
    \item \textbf{Scheduler driving on-sky operations, including with the Feature Based Scheduler}
    \item \textbf{The system can efficiently acquire visits at a given target field.} The median time between
      successive visits for short translational dithers on the angular scale of the ComCam field of view was
      reduced to less than 3 seconds by the end of the campaign\footnote{Because of the \ComCam's small field,
        this intervisit time is consistent with the \c 7s expected for LSSTCam.}, enabling sustained rates of more than 90 individual 30-second visits per hour on a given target field, including occasional filter changes and rotational dithers.
    \item \textbf{ComCam instrument was reliable and effective.}
    \item \textbf{Telescope tracking and boresight stability have been superb.} Analysis of telescope mount encoder data finds a median contribution to the delivered image quality PSF FWHM from tracking jitter of less than 0.01 arcseconds.
    \item \textbf{System optical throughput consistent with a priori expectations.} Measured and predicted counts from per-visit synthetic photometry for the spectrophotometric standard star C26202 are consistent for all bands at the $\sim5\%$ level.
    Comparison to reference catalogs (e.g., from the Dark Energy Survey) shows a tight locus with respect to stellar colors, enabling empirical validation of the color terms for the Rubin Observatory passbands.
    \item \textbf{Data transfer and downstream image analysis software working well.}
    \begin{itemize}
        \item Quicklook campaign at summit supporting nighttime operations
        \item Prompt processing and data release processing campaigns at USDF; running at USDF within a few minutes; Generating (but not publishing) alerts; Finding asteroids; Real-bogus classifier running in alert production pipeline; Minimal impact due to cosmic rays
    \end{itemize}
    \item \textbf{The internal astrometric and photometric calibration are on track to meet design requirements.}
    \begin{itemize}
        \item Stellar astrometric repeatability below 10 mas using only single-frame calibration (i.e., before running global astrometric solution with the gbdes algorithm)
        \item Photometric repeatability for bright stars at level consistent with the statistical Poisson noise photometric errors, with contribution from calibration errors at few mmag level
    \end{itemize}
    \item \textbf{Collimated Beam Projector can co-point with telescope and acquire wavelength scans.}
    \item \textbf{Demonstrated stuttered imaging and guider mode imaging to evaluate high-frequency contributions to delivered image quality (1-100 Hz).}
    \item \textbf{Exercised end-to-end workflows for test planning, execution, and associated data processing and analysis during 7 weeks of continuous operations.}
    \begin{itemize}
        %The effort included summit logistics support for the ComCam on-sky campaign, in-dome daytime testing, ongoing summit engineering, and LSSTCam reverification.
        \item Several hundred team members directly contributed to the ComCam on-sky campaign, supporting continuous 24-hour cycles of daytime and nighttime testing at the summit, observatory operations from the Base Facility in La Serena Chile, data processing campaigns at the US Data Facility at SLAC National Accelerator Laboratory, and data analysis from locations around the world.
        The successes of the ComCam on-sky campaign were made possible by years of preparatory work by the extended Rubin Observatory team.
        \item More than 50 team members worked summit shifts for on-sky testing, and multiple new observing staff were onboarded during the campaign.
        \item The ComCam on-sky campaign required parallel logistics support and coordination for ongoing summit engineering and LSSTCam reverification at the summit.
        \item The team is incorporating lessons learned from the ComCam on-sky campaign towards preparations for LSSTCam on-sky commissioning.
        \item Team is strong.
    \end{itemize}
\end{itemize}

\subsection{Areas of Investigation and Further Development}

At this stage, while system integration and commissioning is ongoing, it is to be expected that the team
would be actively investigating multiple issues.

\begin{itemize}
    \item \textbf{Interpreting stray and scattered light in ComCam observations.} Approximately 20\% of ComCam pointings showed structured patterns of stray light along an edge and/or corner of the ComCam field of view, that are likely attributed to scattered light from bright stars located slightly outside the ComCam field of view.
    Dedicated observations of bright stars were acquired to test the hypothesis that the structured scattered light arises from something related to the ComCam filters, rather than scattering from M2 baffles or another element that will be in place for LSSTCam.
    There is currently no evidence for structured scattered light when the ComCam filters are removed,
    although we expect some features from \c 15-20 degrees off axis as the light/windscreen baffle is
    not yet installed.
    \begin{itemize}
        \item There remains a risk of structured stray light contamination arising from the M2 baffle.
        The concern arises because the black coating on the M2 baffle is anodization rather than a more reflection-suppressing coating such as Aeroglaze.
        There are ongoing modeling efforts and on-sky test planning to better understand the system, and potential stray light paths for LSSTCam.
    \end{itemize}
    \item \textbf{Evaluating contributions to delivered image quality.} Multiple approaches are being pursued to evaluate the various contributions to the delivered image quality during the ComCam on-sky campaign, including the camera, static optics, dynamic optics, mount motion, observatory seeing, and the atmosphere. Observatory environmental controls and monitoring equipment were not yet fully in place during the ComCam campaign.
    \begin{itemize}
        \item Analysis of discrepancies between the estimated AOS residual based on wavefront sensing and
          measurements of the free atmospheric (\eg DIMMs), and delivered image quality
        \item Analysis of ``stuttered''\footnote{data taken while using the CCD's parallel gates to generate
            a series of images of a star.} and guider mode imaging. Preliminary analysis of the stuttered and guider mode imaging data shows strongly correlated common mode motion between stars across the ComCam field of view, suggesting that the delivered image quality during those observations was dominated by dome and/or mirror seeing contributions, or telescope motion.
        \item AOS closed loop convergence optimization
        \item M1M3 thermal contribution and control
        \item Rubin Obs DIMM operations
        \item Dome seeing monitor equipment installation and commissioning
        \item Dome louvers + light and wind screen installation and commissioning
    \end{itemize}
    \item \textbf{Commissioning calibration hardware.}
    \begin{itemize}
        \item Initial commissioning of the Collimated Beam Projector (CBP) using the Simonyi Telescope and
          ComCam took place during the final weeks of the campaing. Analysis of CBP data from ComCam is
          ongoing. Initial analyses shows excesss transmission of red light in the ComCam $g$ filter beyond
          1100 nm, although the impact is not yet clear as the \ComCam CCDs run \c 20C warmer than LSSTCam's.
        \item Flat field screen and illumination system installation and commissioning is ongoing as of January 2025.
    \end{itemize}
    \item \textbf{Increasing overall system reliability and efficiency.} Overall system reliability and efficiency, expressed, for example, in terms of the effective open shutter time fraction, increased throughout the ComCam on-sky campaign. This is an ongoing effort that will continue throughout on-sky commissioning with LSSTCam, and will be continuously monitored and optimized during LSST operations.
    \item \textbf{Improving camera hexapod rotator performance.} The camera hexapod rotator motion performance was sufficient for ComCam on-sky commissioning. The larger LSSTCam field of view requires more stringent performance over the range of telescope altitude and azimuth angles expected for LSST operations.
    \item \textbf{Increasing Simonyi telescope motion performance.} The telescope commanded velocity,
      acceleration, and jerk motion settings were increased during the ComCam on-sky campaign. The telescope is fixed at a horizon pointing orientation during the removal of ComCam and installation of LSSTCam. The ongoing effort to characterize and control inertial forces experienced by the optics with progressively increasing telescope motion settings will resume during the on-sky commissioning campaign with LSSTCam.
\end{itemize}
